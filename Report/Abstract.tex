In this report the simulation and results of a dilute polymer solution is described.  A Monte Carlo simulation in 2D-space is performed to grow polymers segment by segment. The segments only interact with each other through the Lennard-Jones Potential. A Rosenbluth like algorithm is applied to add new segments, favouring lowest energy configurations. To achieve more reliable results a Pruning and Enrichments technique is applied, which promotes lower energy configurations and opposes higher energy configurations. This keeps the weights of the polymers from fluctuating too much, which reduces variance. The simulation is used to calculate polymer properties of 2D configurations like the End-to-end distance and the Gyradius. A total of 10000 and 5000 polymers are grown to a length of 350 segments, using the Rosenbluth algorithm and the PERM algorithm respectively.