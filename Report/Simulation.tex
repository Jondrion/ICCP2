In the simulation 10000 polymers were grown. Each beat of the polymer has a reduced length of 1.0 with respect to the previous beat. The acting Lennart-Jones potential uses reduced units of $\epsilon=0.25$ and $\sigma=0.8$, also the reduced temperature is set to $T=1.0$. All the polymers are grown to a length $N$ of 300. 

We are interested in the average End-to-end-distance $R$ as a function of polymer length $N$ and in the average Gyradius as a function of length $N$. The average End-to-end-distance $R$ will be compared to the theoretical result, according to which it should relate to $N$ as $R\propto N^{0.75}$.

In order to leave out polymers which have become trapped and are therefore forced to overcross themselves, we apply pruning and enriching. The limits for pruning and enriching are determined and dynamically updated according to the average weight of each polymer for each length $N$. This method ensures only polymers which haven't trapped themselves are only considered in the results.