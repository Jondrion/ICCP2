In a dilute polymer solution we can study the effect of a solvent by looking at ensembles of individual polymers. A good solvent means that the free energy of a polymer segment that is surrounded by solvent will be less than that of a segment which is close to other segments in the polymer. By introducing an effective repulsion between polymer segments we achieve the same effect.

The model aimed for is one that preserves the important properties of a polymer without incorporating all the atomic details of its structure. Polymers are large molecules in which a number of repeated segments form long chains. The number of such segments $N$ is typically $10^3 - 10^5$. In this report we will be working on the large scale of the polymer and we will therefore make the simplification of universal behaviour. This means that the properties of the polymer are not dependent on the chemical nature of it and we can look at the segments as closed units or ``beads''. Neighbouring beads have a fixed mutual distance and no further interaction. Furthermore, the beads repel when they overlap and experience an attractive van der Waals' interaction on longer distances. One last interaction is the effective repulsion between segments due to the solvent effect, described earlier. The interaction of non-neighbouring beads can therefore be described by a Lennard-Jones interaction. Because the beads repel each other the polymer represents a self-avoiding walk or SAW -- as opposed to the random walk.

The polymers in this report are studied in an off-lattice model in two dimensions. In the next sections this model is implemented in an algorithm.