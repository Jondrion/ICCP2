The study of polymers is important for its wide uses in industry and everyday life. Polymers are composed of an integer number of unit-segments, or monomers, which together form a polymer. When ensembling a polymer by repeating unit segments it can take all sorts of configurations, depending on the angle between subsequent segments. This angle is determined under the influence of the solvent of the polymer and interaction forces between the molecules. 

Such a polymer can be modelled by representing each segment as a ``bead'' which has a fixed distance to the previous bead. Depending on the polymer and the solvent a force, which acts between the beads, can be introduced. In this research a dilute polymer solution in 2D is simulated and the force between beads is modelled as a Lennard-Jones force.

A common method used for simulating polymers like this is the Rosenbluth algorithm, which adds subsequent beads trying to avoid high energy configurations. In this research we will make use of a Rosenbluth like algorithm which makes additional use of a pruning and enrichment strategy to build polymers. This strategy promotes low energy configurations and kills high energy configurations. 

Interesting properties, which are studied, are the average End-to-end distance as a function of length and how compact the polymer becomes as a function of length, this can be expressed in terms of the radius of gyration, or Gyradius.  